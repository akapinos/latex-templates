\documentclass[oneside,final,14pt]{extreport}
\usepackage[utf8]{inputenc}
\usepackage{tikz}
\usepackage[russian]{babel}
\usepackage{vmargin}
\usepackage{multirow}
\usepackage[resetfonts]{cmap}
\usepackage{caption}
\setpapersize{A4}
\setmarginsrb{2cm}{1.5cm}{1cm}{1.5cm}{0pt}{0mm}{0pt}{13mm}
\usepackage{indentfirst}
\sloppy

\newenvironment{compactlist}{
\begin{list}{{$\bullet$}}{
  \setlength\partopsep{0pt}
  \setlength\parskip{0pt}
  \setlength\parsep{0pt}
  \setlength\topsep{0pt}
  \setlength\itemsep{0pt}
  }
}{
  \end{list}
}

\newenvironment{letterlist}{
\newcounter{lcounter}
\begin{list}{\alph{lcounter})}{\usecounter{lcounter}}{
  \setlength\partopsep{0pt}
  \setlength\parskip{0pt}
  \setlength\parsep{0pt}
  \setlength\topsep{0pt}
  \setlength\itemsep{0pt}
  }
}{
  \end{list}
}

\DeclareCaptionFormat{GOSTtable}{#2#1\\#3}
\DeclareCaptionLabelSeparator{fill}{\hfill}
\DeclareCaptionLabelFormat{fullparents}{{#1}{~}#2}

\captionsetup[table]{
format=GOSTtable,
font={footnotesize},
labelformat=fullparents,
labelsep=fill,
labelfont=it,
textfont=bf,
justification=centering,
singlelinecheck=false
}

\newcommand{\arrowcircle}[1][]{
\begin{tikzpicture}[#1]

\draw[->] (0, 0ex) -- (1em, 0ex);
\draw (1em,0ex) circle (1.2ex);

\end{tikzpicture}
}

\newcommand{\Section}[1]{
\refstepcounter{section}
\section*{\centering #1}
\addcontentsline{toc}{section}{#1}
}

\newcommand{\Subsection}[1]{
\refstepcounter{subsection}
\subsection*{\centering #1}
\addcontentsline{toc}{subsection}{#1}
}

\newcommand{\Subsubsection}[1]{
\refstepcounter{subsubsection}
\subsubsection*{\centering #1}
\addcontentsline{toc}{subsubsection}{#1}
}

\begin{document}

\begin{titlepage}

\centerline{БЕЛОРУССКИЙ ГОСУДАРСТВЕННЫЙ УНИВЕРСИТЕТ}
\centerline{ФАКУЛЬТЕТ РАДИОФИЗИКИ И КОМПЬЮТЕРНЫХ ТЕХНОЛОГИЙ}
\centerline{\hfill
\hrulefill \hrulefill \hrulefill \hrulefill \hrulefill \hrulefill
\hfill}

\vfill
\vfill
\vfill
\vfill
\vfill
\vfill

\Large
\centerline{\bf Лабораторная работа №}
\begin{centering}

{\bf }

\end{centering}

\vfill
\vfill
\vfill

\normalsize
\begin{flushright}

Выполнил \\
студент 2 курса 4 группы \\
Капинос Алексей \\
Преподаватель \\
Прокопович И. П.
\end{flushright}

\vfill
\vfill

\centerline{Минск~--- 2020}

\end{titlepage}
\setcounter{page}{2}

{\noindent \bf Цель работы:}%

\Section{Краткая теория}

\Section{Практическая часть}

\clearpage

{\noindent\bf Вывод:}%

\end{document}
